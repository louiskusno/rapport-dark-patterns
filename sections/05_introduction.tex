\section{Introduction}

L'économie numérique s'appuie massivement sur le profilage des utilisateurs et la personnalisation algorithmique des contenus. En théorie, cette personnalisation vise à améliorer l'expérience utilisateur en proposant des produits, services ou informations adaptés aux préférences individuelles. Cependant, dans la pratique, ces systèmes de recommandation fonctionnent comme des « boîtes noires » : ni les utilisateurs ni les régulateurs ne savent précisément quelles données personnelles alimentent ces profils, ni selon quels critères les décisions d'affichage sont prises.

Ce manque de transparence crée un terrain propice aux pratiques abusives. On observe ainsi des phénomènes de discrimination de prix (un même produit affiché à des tarifs différents selon le profil utilisateur), ou de recherche (des résultats ordonnés différemment pour orienter les choix d'achat). S'ajoutent à cela les « dark patterns », des mécanismes d'interface conçus pour exploiter les biais cognitifs humains : faux compteurs de stock créant un sentiment d'urgence, promotions mensongères, ou encore boutons trompeurs compliquant la désinscription.
