\section{Solution Proposée : STETOSCOPE}

\subsection{Vue d'ensemble de l'architecture}

STETOSCOPE est une plateforme composée d'une application mobile (pour les participants) et d'un serveur backend (pour l'administration et l'analyse). Le fonctionnement repose sur quatre étapes clés :

\begin{enumerate}
    \item \textbf{Configuration de la campagne :} L'administrateur définit un scénario, les URL cibles et les instructions de guidage.
    
    \item \textbf{Guidage de l'utilisateur :} Le participant sélectionne une campagne dans l'application. Une bannière en surimpression (overlay) le guide pas à pas dans l'application tierce (ex : Amazon) jusqu'à l'information cible.
    
    \item \textbf{Capture :} L'utilisateur déclenche manuellement une capture d'écran. Celle-ci est envoyée au serveur avec des métadonnées (modèle du téléphone, localisation si autorisée).
    
    \item \textbf{Analyse automatique :} Le serveur traite les captures pour extraire les données pertinentes (prix, rang des produits) et générer des visualisations.
\end{enumerate}

\subsection{Implémentation Technique}

\begin{itemize}
    \item \textbf{Application Mobile :} Développée en Java pour Android. Elle nécessite des permissions spécifiques pour s'afficher « par-dessus » les autres applications (overlay) afin de guider l'utilisateur sans quitter le contexte de navigation.
    
    \item \textbf{Backend et Dashboard :} L'interface d'administration est construite avec le framework Vite (React/TypeScript) pour le frontend et FastAPI pour l'API REST.
    
    \item \textbf{Traitement des données :} L'outil permet d'extraire le texte des images via OCR et d'appliquer des expressions régulières (Regex) pour structurer les données. Dans le cadre de ce P-SAT, nous explorons également l'utilisation de LLM (Large Language Models) pour améliorer cette extraction sur des données complexes.
\end{itemize}
