\section{La plateforme STETOSCOPE}

\subsection{Vue d'ensemble de l'architecture}

STETOSCOPE est une plateforme de crowdsourcing initialement développée pour auditer la transparence des plateformes numériques sur Android. Elle se compose d'une application mobile pour les participants et d'un serveur backend pour l'administration.

\begin{figure}[htbp]
    \centering
    \includegraphics[width=0.8\textwidth]{images/stetoscope workflow.jpg}
    \caption{Architecture de STETOSCOPE : 1) Campagne configurée, 2) Guidage de l'utilisateur, 3) Capture d'écran, 4) Traitement automatique.}
    \label{fig:workflow}
\end{figure}

Le fonctionnement repose sur quatre étapes clés :

\begin{enumerate}
    \item \textbf{Configuration de la campagne :} L'administrateur définit un scénario via l'interface web.

    \item \textbf{Guidage de l'utilisateur :} Le participant sélectionne une campagne dans l'application mobile (voir Figure \ref{fig:mobile_app}). Une bannière en surimpression le guide jusqu'à l'information cible.

    \item \textbf{Capture :} L'utilisateur déclenche manuellement une capture d'écran envoyée au serveur.

    \item \textbf{Analyse automatique :} Le serveur traite les captures pour extraire les données.
\end{enumerate}

\begin{figure}[htbp]
    \centering
    \begin{minipage}{0.45\textwidth}
        \centering
        \includegraphics[width=\textwidth]{images/stetoscope screenshot.jpg}
        \caption{Interface de l'application mobile listant les campagnes.}
        \label{fig:mobile_app}
    \end{minipage}
    \hfill
    \begin{minipage}{0.45\textwidth}
        \centering
        \includegraphics[width=\textwidth]{images/stetoscope admin dashbord.jpg}
        \caption{Tableau de bord de l'administrateur pour la gestion et l'analyse.}
        \label{fig:admin_dashboard}
    \end{minipage}
\end{figure}

\subsection{Limites initiales}

Bien que fonctionnel, l'outil présentait plusieurs limitations majeures avant notre intervention :
\begin{itemize}
    \item \textbf{Extraction rigide :} L'extraction des données (prix, noms de produits) reposait exclusivement sur l'OCR couplé à des expressions régulières (Regex), limitant la flexibilité face à des mises à jour d'interfaces.
    \item \textbf{Écosystème restreint :} L'application de collecte n'était disponible que sur Android, excluant de fait les utilisateurs d'iOS et les environnements de bureau.
    \item \textbf{Visualisation basique :} Le tableau de bord intégré ne permettait pas d'analyses croisées complexes ou de filtrage dynamique avancé.
\end{itemize}
