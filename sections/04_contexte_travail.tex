\section*{Contexte de travail}

Ce projet s'inscrit dans le cadre du cours P-SAT (Projet Scientifique, Artistique et Technique) sous la direction d'Antoine Boutet, enseignant-chercheur au département Informatique de l'INSA Lyon. L'équipe projet est composée de trois étudiants de 5\textsuperscript{e} année du département IF.

\textbf{Collaboration interdisciplinaire :} Le projet bénéficie d'une collaboration enrichissante avec le département Télécommunications, Services et Usages (TC). Isabelle Ott Kiraly (\texttt{isabelle.ott-kiraly@insa-lyon.fr}) et Andy Vu Ngoc (\texttt{andy.vu-ngoc@insa-lyon.fr}) contribuent activement à la rédaction des protocoles d'étude sur STETOSCOPE, à l'organisation des campagnes de collecte de données, ainsi qu'à l'analyse bibliographique et à l'état de l'art.

\textbf{Support technique et ressources :} Le développement s'appuie sur les travaux initiaux de Tao Beaufils (\texttt{tao.beaufils@inria.fr}), développeur principal du logiciel STETOSCOPE lors de son Projet de Fin d'Études (PFE). Des échanges réguliers avec lui permettent de mieux comprendre l'architecture logicielle existante et de planifier les évolutions futures. Pour les phases de développement et de test, deux smartphones Android sont mis à disposition de l'équipe. Le suivi du projet est assuré par des réunions hebdomadaires avec l'encadrant, garantissant un accompagnement continu et une réactivité face aux difficultés rencontrées.
