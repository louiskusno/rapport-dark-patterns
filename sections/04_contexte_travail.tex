\section*{Contexte de travail}

Ce projet s'inscrit dans le cadre du cours P-SAT (Projet Scientifique, Artistique et Technique) sous la direction d'Antoine Boutet, enseignant-chercheur au département Informatique de l'INSA Lyon et dans le groupe Inria Privatics. L'équipe projet est composée de trois étudiants de 5\textsuperscript{e} année du département informatique.

\textbf{Collaboration interdisciplinaire :} Jusqu'à début décembre, le projet a été réalisé en collaboration avec 2 étudiants étrangers du département Télécommunications : Isabelle Ott Kiraly et Andy Vu Ngoc. Ils ont contribué à la rédaction des protocoles d'étude sur Stetoscope, à la communication de la campagne de collecte de données, ainsi qu'à l'analyse bibliographique.

\textbf{Support technique et ressources :} Tao Beaufils (\texttt{tao.beaufils@inria.fr}) s'est occupé du développement de Stetoscope pendant son Projet de Fin d'Études (PFE). Des échanges avec lui nous ont permis de mieux comprendre l'architecture logicielle existante et de planifier les évolutions futures. Pour les phases de développement et de test, deux smartphones Android nous ont été mis à disposition. Le suivi du projet est assuré par des réunions hebdomadaires avec Antoine, garantissant un accompagnement continu.
