\section{Analyse et Discussion}

\subsection{Perception des utilisateurs}

Pendant la récolte de données organisée mi-janvier, nous avons d'abord interrogé les étudiants sur leur perception de ces pratiques. Parmi eux, 88\% avaient déjà été témoins d'une discrimination de prix, notamment l'augmentation de leurs billets de transport après une première consultation. Il est cependant difficile de savoir si cette augmentation était strictement personnel ou si elle avait eu lieu pour tous les utilisateurs. La majorité d'entre eux étaient conscients d'obtenir des résultats  différents ou similaires pour une même recherche en fonction de la plateforme utilisée. Ils ont qualifiés les compteurs comme utiles et incitants à l'achat, mais parfois aussi malhonnêtes. Enfin, leurs réactions face aux réductions du Black Friday étaient plutôt partagées (voir ci dessous Figure \ref{fig:wooclap}) : 

\begin{figure}[!ht]
    \centering
    \includegraphics[width=0.7\textwidth]{images/resultats_wooclap.png}
    \caption{Sondage réalisé auprès d'étudiants de l'INSA le 12/01/2026}
    \label{fig:wooclap}
\end{figure}

\subsection{Apports du projet}

L'approche participative adoptée par STETOSCOPE prouve sa pertinence pour auditer les algorithmes de personnalisation, particulièrement dans l'écosystème mobile souvent opaque. En impliquant directement l'utilisateur final, nous parvenons à contourner les systèmes de protection anti-bots et à collecter des données authentiques reflétant l'expérience réelle des consommateurs.

Nos contributions techniques ont apporté une valeur ajoutée significative à cette démarche. D'une part, l'introduction des modèles de langage de grande taille (LLM) pour l'extraction automatique des données a considérablement accru la flexibilité du système, permettant d'analyser des captures d'écran visuellement complexes là où les méthodes traditionnelles par Regex échouaient. D'autre part, le passage à Power BI a transfiguré l'exploitation de ces données. Cet outil permet de déceler avec une précision accrue des tendances de personnalisation jusque-là invisibles, en croisant par exemple les variations tarifaires avec la marque du terminal ou la localisation géographique précise de l'utilisateur.

\subsection{Limitations et Éthique}

Bien que nous puissions observer avec précision les résultats de la personnalisation (l'output), le système algorithmique des plateformes reste une boîte noire quant aux mécanismes décisionnels internes. De plus, la collecte de données privées via des captures d'écran a nécessité une attention particulière pour garantir une conformité totale avec le RGPD. Des protocoles de floutage automatique ont été mis en place pour protéger les informations sensibles des participants, sous la supervision du DPO (Délégué à la Protection des Données).
