\section{Conclusion}

Ce projet P-SAT a démontré la puissance du crowdsourcing pour lever le voile sur les pratiques opaques des plateformes numériques. En s'appuyant sur STETOSCOPE, nous avons pu collecter et analyser des preuves empiriques de manipulations algorithmiques et de dark patterns.

Nos travaux ont permis de transformer une plateforme de collecte Android initiale en un écosystème d'audit multi-plateforme performant. L'intégration de modèles de vision LLM et de tableaux de bord Power BI a radicalement amélioré la qualité et la profondeur de l'extraction et de l'analyse des données. Les campagnes menées ont confirmé des pratiques préoccupantes : discriminations tarifaires sur AliExpress, biais de classement sur Amazon, et promotions artificielles sur Temu.

À terme, ces outils pourraient être mis à disposition d'observatoires de la transparence numérique ou de régulateurs pour assurer une protection accrue des consommateurs face aux dérives de la personnalisation algorithmique.
