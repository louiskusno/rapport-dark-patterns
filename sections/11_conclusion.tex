\section{Conclusion}

Ce projet P-SAT a permis de démontrer la viabilité et l'efficacité de l'approche crowdsourcing pour l'audit des algorithmes de personnalisation et la détection de dark patterns. STETOSCOPE offre une alternative crédible aux méthodes automatisées traditionnelles en intégrant l'utilisateur réel dans le processus de collecte de données.

Les campagnes menées ont révélé des preuves concrètes de pratiques problématiques : discriminations tarifaires sur AliExpress (variations de prix jusqu'à 183\%), orientations biaisées des résultats de recherche sur Amazon (écarts moyens de prix allant de 235€ à 1100€), et promotions trompeuses sur Temu. Ces résultats soulignent l'importance d'une surveillance continue des plateformes numériques.

Les travaux futurs viseront à étendre la plateforme aux écosystèmes iOS et Web, à améliorer l'extraction automatique de données via des modèles de vision LLM, et à faciliter l'analyse des résultats grâce à des outils de visualisation comme Power BI. L'objectif à long terme est de rendre STETOSCOPE accessible au grand public pour une meilleure transparence algorithmique.
