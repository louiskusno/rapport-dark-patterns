\section{Campagnes de Collecte de Données}

La collecte des données a été structurée autour de trois approches complémentaires, permettant d'obtenir un jeu de données à la fois régulier et ponctuellement riche.

\subsection{Méthodologies de collecte}

\begin{enumerate}
    \item \textbf{Collecte quotidienne :} Les trois membres de l'équipe ont utilisé quotidiennement l'application STETOSCOPE pour signaler des cas de personnalisation des prix rencontrés lors de leur navigation personnelle.
    
    \item \textbf{Suivi saisonnier et événementiel :} Une attention particulière a été portée aux périodes de fortes promotions, notamment le \textbf{Black Friday} et les fêtes de \textbf{Noël}. Durant ces phases, la collecte s'est focalisée sur les articles porteurs de labels « promotionnels » afin de vérifier l'authenticité des rabais.
    
    \item \textbf{Atelier participatif à l'INSA (12 janvier 2026) :} Organisée en collaboration avec Antoine BOUTET, cette campagne d'envergure a mobilisé plus de vingt participants. 
    \begin{itemize}
        \item \textbf{Déroulement :} Après une présentation pédagogique sur les dark patterns et la personnalisation algorithmique (enrichie par des jeux Wooclap), les participants ont été invités à installer STETOSCOPE.
        \item \textbf{Incitation :} Une distribution de pizzas gratuites a favorisé l'engagement et la convivialité.
        \item \textbf{Impact :} Cette session a permis de générer un pic important de données synchronisées sur des requêtes identiques, facilitant la comparaison directe entre profils.
    \end{itemize}
\end{enumerate}

\begin{figure}[htbp]
    \centering
    \includegraphics[width=0.6\textwidth]{images/poster.png}
    \caption{Affiche de la campagne de collecte organisée à l'INSA Lyon.}
    \label{fig:poster}
\end{figure}
