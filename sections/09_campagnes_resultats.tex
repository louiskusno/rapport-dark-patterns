\section{Campagnes de Collecte de Données}

La collecte des données a été structurée autour de trois approches complémentaires, permettant d'obtenir un jeu de données à la fois régulier et ponctuellement riche.

\subsection{Méthodologies de collecte}

La stratégie de collecte des données a été structurée autour de trois approches complémentaires, permettant de constituer un jeu de données à la fois régulier sur le long terme et ponctuellement riche lors d'événements spécifiques.

La première approche repose sur une collecte quotidienne assurée par les trois membres de l'équipe projet. En utilisant l'application STETOSCOPE au fil de leur navigation personnelle, ils ont pu documenter de nombreux cas de personnalisation des prix rencontrés "en conditions réelles". Cette source de données continue a été complétée par un suivi saisonnier et événementiel rigoureux, ciblant particulièrement les périodes de fortes promotions telles que le Black Friday et les fêtes de Noël. Durant ces phases critiques, l'effort de collecte s'est focalisé sur les articles arborant des labels promotionnels afin d'auditer la véracité des rabais affichés.

Enfin, une campagne d'envergure sous forme d'atelier participatif a été organisée à l'INSA Lyon le 12 janvier 2026, en collaboration étroite avec Antoine BOUTET. Cet événement a réuni plus de vingt participants autour d'une présentation pédagogique traitant des dark patterns et des enjeux de la personnalisation algorithmique, ponctuée de sessions interactives via l'outil Wooclap. Pour encourager la participation et garantir un climat convivial, une distribution de pizzas gratuites a été organisée. Sur le plan scientifique, cet atelier a été crucial car il a permis de générer un pic massif de données synchronisées : plusieurs utilisateurs effectuant simultanément les mêmes requêtes, ce qui a grandement facilité la comparaison directe des profils et l'identification des biais algorithmiques (voir Figure \ref{fig:poster}).

\begin{figure}[htbp]
    \centering
    \includegraphics[width=0.6\textwidth]{images/poster.png}
    \caption{Affiche de la campagne de collecte organisée à l'INSA Lyon.}
    \label{fig:poster}
\end{figure}
