\section{Discussion}

\subsection{Apports du projet}

STETOSCOPE démontre qu'une approche participative est viable et nécessaire pour auditer les algorithmes de personnalisation mobile. En impliquant l'utilisateur, nous contournons les défenses anti-bots et collectons des données authentiques (« User-centric »). L'outil a permis de mettre en évidence des preuves tangibles de discriminations tarifaires et de fausses promotions.

\subsection{Limitations et Éthique}

L'approche présente des défis. Bien que nous puissions observer le résultat de la personnalisation (l'output), le système reste une boîte noire quant aux critères exacts utilisés (l'input). De plus, la collecte de captures d'écran, bien que floutées pour les parties non pertinentes, soulève des enjeux de confidentialité qui ont été validés par le DPO (Délégué à la Protection des Données).
