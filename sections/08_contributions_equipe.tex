\section{Amélioration de Stetoscope : Nos Contributions}

Afin de pallier les limitations précédemment identifiées, nous avons développé trois axes d'amélioration.

\begin{figure}[!ht]
    \centering
    \includegraphics[width=0.9\textwidth]{images/workflow.png}
    \caption{Pipeline de traitement des données : de la collecte via Stetoscope à la visualisation Power BI.}
    \label{fig:workflow}
\end{figure}

\subsection{Extension multi-plateforme (Web)}

Afin d'étendre les collectes au-delà des appareils Android, nous avons développé une version Web de Stetoscope. Sur le plan technologique, cette extension interagit directement avec les API backend existantes, permettant ainsi de soumettre des captures d'écran et des métadonnées depuis n'importe quel navigateur moderne. L'impact de cette évolution est significatif, puisqu'il est maintenant possible de participer aux campagnes depuis un iOS ou un ordinateur. Par conséquent, nous sommes capables d'intégrer n'importe quel utilisateur et de diversifier les appareils étudiés(voir Figures \ref{fig:web_list} et \ref{fig:web_interface}).

\begin{figure}[!ht]
    \centering
    \begin{minipage}{0.45\textwidth}
        \centering
        \includegraphics[width=\textwidth]{images/stetoscope web1.png}
        \caption{Interface de la liste des tâches de la version Web de STETOSCOPE.}
        \label{fig:web_list}
    \end{minipage}
    \hfill
    \begin{minipage}{0.45\textwidth}
        \centering
        \includegraphics[width=\textwidth]{images/stetoscope web2.png}
        \caption{Interface détaillée d'une tâche sur la version Web (exemple AliExpress).}
        \label{fig:web_interface}
    \end{minipage}
\end{figure}

\subsection{Analyse avancée par LLM}

L'extraction de données par Regex a très vite été insuffisante pour exploiter les résultats, donc nous avons implémenté un pipeline indépendant de Stetoscope qui exploite les grands modèles de langage (LLM). Cette implémentation repose sur un script Python qui transmet les captures d'écran à l'API d'OpenAI. Nous avons fait attention au prompt utilisé pour que le modèle soit capable d'extraire des informations de manière sémantique. Contrairement à l'approche par Regex, le LLM peut traiter des cas complexes tels que l'identification simultanée de plusieurs produits sur une même capture ou l'analyse de l'ordre d'apparition des résultats. En sortie, les données sont structurées sous format JSON avant d'être converties en CSV pour faciliter les phases ultérieures d'analyse statistique.

\subsection{Visualisation et Analyse via Power BI}

Pour transformer les données brutes récoltées, nous avons intégré Power BI comme outil de visualisation. Nous avons choisi cette solution car elle offre de très bonnes capacités de filtrage et d'agrégation, et que nous avions jamais eu l'occasion de l'utiliser. Nous avons donc pu réaliser différents graphiques adaptés aux campagnes pour visualiser les variations de prix selon l'utilisateur ou la période temporelle par exemple. (voir Figure \ref{fig:powerbi}).

\begin{figure}[!ht]
    \centering
    \includegraphics[width=0.8\textwidth]{images/power-bi.pdf}
    \caption{Tableau de bord Power BI présentant l'analyse de la personnalisation des prix sur AliExpress.}
    \label{fig:powerbi}
\end{figure}

