\section{Amélioration de STETOSCOPE : Nos Contributions}

Afin de pallier les limitations précédemment identifiées, nous avons développé trois axes d'amélioration majeurs qui constituent le cœur de notre apport au projet.

\subsection{Extension multi-plateforme (Web)}

Pour étendre la portée de la collecte au-delà de l'écosystème Android, nous avons développé une version Web de STETOSCOPE.
\begin{itemize}
    \item \textbf{Technologie :} En interagissant directement avec les API backend existantes, cette extension permet de soumettre des captures d'écran depuis n'importe quel navigateur.
    \item \textbf{Impact :} Cela rend la collecte possible sur iOS et PC, diversifiant les profils utilisateurs (voir Figures \ref{fig:web_list} et \ref{fig:web_interface}).
\end{itemize}

\begin{figure}[htbp]
    \centering
    \includegraphics[width=0.7\textwidth]{images/stetoscope web1.png}
    \caption{Interface de la liste des tâches de la version Web de STETOSCOPE.}
    \label{fig:web_list}
\end{figure}

\begin{figure}[htbp]
    \centering
    \includegraphics[width=0.7\textwidth]{images/stetoscope web2.png}
    \caption{Interface détaillée d'une tâche sur la version Web (exemple AliExpress).}
    \label{fig:web_interface}
\end{figure}

\subsection{Analyse avancée par LLM}

L'extraction de données par Regex est limitée par la structure changeante des sites e-commerce. Nous avons donc remplacé cette approche par un pipeline utilisant des modèles de langage de grande taille (LLM).
\begin{itemize}
    \item \textbf{Fonctionnement :} Un script Python envoie les captures d'écran à l'API d'OpenAI (modèles vision). Grâce à un prompt-engineering précis, le modèle extrait les informations de manière sémantique.
    \item \textbf{Flexibilité :} Contrairement aux Regex, le LLM peut traiter des cas complexes comme l'identification de plusieurs produits dans une seule capture, l'extraction de l'ordre d'apparition des résultats, ou la distinction entre un produit organique et une publicité.
    \item \textbf{Sortie structurée :} Les résultats sont fournis sous format JSON structuré, puis convertis en CSV pour analyse ultérieure.
\end{itemize}

\subsection{Visualisation et Analyse via Power BI}

Pour transformer les données brutes en informations exploitables, nous avons intégré Power BI comme outil de visualisation principal.
\begin{itemize}
    \item \textbf{Capacités d'analyse :} Power BI offre des capacités de filtrage et d'agrégation bien supérieures au dashboard d'origine. Il permet par exemple d'agréger les prix par modèle de smartphone, de filtrer par date ou par gamme de prix.
    \item \textbf{Interactivité :} Les graphiques interactifs permettent de mettre rapidement en évidence les corrélations (ou l'absence de corrélation) entre le profil utilisateur et les variations de prix ou de classement.
\end{itemize}
