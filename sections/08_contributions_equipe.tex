\section{Amélioration de STETOSCOPE : Nos Contributions}

Afin de pallier les limitations précédemment identifiées, nous avons développé trois axes d'amélioration majeurs qui constituent le cœur de notre apport au projet.

\subsection{Extension multi-plateforme (Web)}

Afin d'étendre la portée de la collecte au-delà de l'écosystème Android, nous avons développé une version Web de STETOSCOPE. Sur le plan technologique, cette extension interagit directement avec les API backend existantes, permettant ainsi de soumettre des captures d'écran et des métadonnées depuis n'importe quel navigateur moderne. L'impact de cette évolution est significatif, puisqu'elle rend la collecte possible sur des plateformes jusqu'alors inaccessibles comme iOS ou les ordinateurs de bureau. Cette versatilité permet de diversifier considérablement les profils utilisateurs et d'enrichir les environnements d'audit (voir Figures \ref{fig:web_list} et \ref{fig:web_interface}).

\begin{figure}[!ht]
    \centering
    \begin{minipage}{0.45\textwidth}
        \centering
        \includegraphics[width=\textwidth]{images/stetoscope web1.png}
        \caption{Interface de la liste des tâches de la version Web de STETOSCOPE.}
        \label{fig:web_list}
    \end{minipage}
    \hfill
    \begin{minipage}{0.45\textwidth}
        \centering
        \includegraphics[width=\textwidth]{images/stetoscope web2.png}
        \caption{Interface détaillée d'une tâche sur la version Web (exemple AliExpress).}
        \label{fig:web_interface}
    \end{minipage}
\end{figure}

\subsection{Analyse avancée par LLM}

L'extraction de données par Regex montrant ses limites face à la versatilité des interfaces de e-commerce, nous avons implémenté un pipeline innovant exploitant les modèles de langage de grande taille (LLM). Le coeur du dispositif repose sur un script Python qui transmet les captures d'écran à l'API d'OpenAI, en utilisant des modèles dotés de capacités de vision. Grâce à une ingénierie de prompt rigoureuse, le modèle est capable d'extraire des informations de manière sémantique. Contrairement à l'approche par Regex, le LLM peut traiter des cas complexes tels que l'identification simultanée de plusieurs produits sur une même capture, l'analyse de l'ordre d'apparition des résultats, ou encore la distinction critique entre les résultats de recherche organiques et les publicités sponsorisées. En sortie, les données sont structurées sous format JSON avant d'être converties en CSV pour faciliter les phases ultérieures d'analyse statistique.

\subsection{Visualisation et Analyse via Power BI}

Pour transformer la masse de données brutes récoltée en informations stratégiques, nous avons intégré Power BI comme outil pivot de visualisation. Cette solution offre des capacités de filtrage et d'agrégation bien supérieures au tableau de bord d'origine. Désormais, il est possible d'agréger les variations de prix en fonction de critères précis tels que le modèle de smartphone, la localisation géographique ou la période temporelle. L'interactivité des graphiques Power BI permet de mettre rapidement en évidence des corrélations complexes et de documenter de manière visuelle et indiscutable les pratiques de personnalisation algorithmique identifiées au cours du projet.
